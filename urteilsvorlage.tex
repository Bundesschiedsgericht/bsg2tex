%
%   Beispieldokument für bsg2tex, Stand 08. August 2013
%   CC-BY Markus Gerstel
%

%\documentclass{bsg2tex}
%\documentclass[anonym]{bsg2tex}
%\documentclass[print]{bsg2tex}
%\documentclass[invprint]{bsg2tex}

% !TEX encoding = UTF-8 Unicode
%
% bsg2tex - Bundesschiedsgericht
%
% Bundesschiedsgerichts-spezifische Anpassung von Seitenelementen
% Stand: 29.08.2014
%
% Andere Schiedsgerichte sollten diese Datei als Vorlage nehmen, und
% ihre eigene landesschiedsgericht.tex bauen und \include{}'n.
%

%
% Textblock oben rechts: Name, Anschrift, Mailadresse, Ort, Datum, Aktenzeichen
% Wenn die Größe der \parbox geändert werden muss, wird eventuell eine Anpassung
% der Koordinaten und des freigestellten Bereichs im Template notwendig.
%
\newcommand{\Header}{
\parbox[t][36mm]{60mm}{\small%
Piratenpartei Deutschland\\
Bundesschiedsgericht\\
Pflugstraße 9a, 10115 Berlin\\
schiedsgericht@piratenpartei.de\\[0.1em]
Berlin, den \textbf{\Datum}\\
AZ: \textbf{\Aktenzeichen}}
}

%
% Einleitungszeile des Footers
%
\newcommand{\RichterTabelleHeader}{%
\small\centering\textcolor{white}{%
Das Bundesschiedsgericht der Piratenpartei Deutschland wird vertreten durch:%
}%
}

%
% Der eigentliche Footer
% Eine 8-spaltige Tabelle (Anzahl 'c's in der Zeile mit 'tabular').
% In der einzeiligen Tabelle wird dann jede Spalte grundsätzlich auf eine spezifische Breite (XXmm) forciert.
% Für ein 7-köpfiges Gremium werden 26mm empfohlen, wobei längere Namen oder Titel ausnahmsweise
% breitere Spalten notwendig machen können ('Vorsitzender~Richter' bspw. 26.5mm).
% Für ein 6-köpfiges oder kleineres Gremium werden 28mm empfohlen.
% 
% Alle \parbox-Zeilen ausser der letzten müssen mit &% enden, und es sollten in dem ganzen Makro
% keine überflüssigen Leerstellen oder Returns vorkommen.
%
% Soll die Tabelle insgesamt horizontal ausgebreitet werden, kann \tabcolsep (Spaltenabstand)
% von 0mm auf einen höheren Wert gestellt werden. Einfach ausprobieren.
%
\newcommand{\RichterTabelle}{%
\setlength{\tabcolsep}{0mm}%
\centering\textcolor{white}{%
\large\begin{tabular}{ccccccc}%
\parbox[t]{26.5mm}{\centering Georg\\v. Boroviczeny\\[0.2em]\footnotesize{Ersatzrichter}}&%
\parbox[t]{26.5mm}{\centering Gregory\\Engels\\[0.2em]\footnotesize{Richter}}&%
\parbox[t]{26.5mm}{\centering Mario\\Longobardi\\[0.2em]\footnotesize{Richter}}&%
\parbox[t]{26.5mm}{\centering Michael\\Ebner\\[0.2em]\footnotesize{Vorsitzender~Richter}}&%
\parbox[t]{26.5mm}{\centering Klaus\\Sommerfeld\\[0.2em]\footnotesize{Richter}}&%
\parbox[t]{26.5mm}{\centering Holger\\van Lengerich\\[0.2em]\footnotesize{Richter}}&%
\parbox[t]{26.5mm}{\centering Stefan\\Thöni\\[0.2em]\footnotesize{Ersatzrichter}}%
\end{tabular}}%
}

\newcommand{\Aktenzeichen}{\textcolor{red}{AZ fehlt!}}
\newcommand{\Datum}       {\textcolor{red}{Datum fehlt!}}
\newcommand{\Titel}       {\textcolor{red}{Urteil/Beschluss} zu {\Aktenzeichen}}

\begin{document}

In dem Verfahren \Aktenzeichen\\

1. \Anonym{Marianne Herpsen, Derpstr. 3, 0815 Umseck,}\\[0.1em]
2. \Anonym{Heinz-Rüdiger Herpsen, Derpstr. 3, 0815 Umseck,}\\[0.1em]
vertreten durch \Anonym{Dr. Rodriguez Rechtsstaat, Anwaltsstr. 6, 31337 Schaffhausen, art20@example.com,}\\[0.1em]
--- Antragsteller und Berufungsgegner ---

gegen

Piratenpartei, Landesverband Salzburg, \Anonym{Schnitzelstr. 4-8, 04711 Baumhaus,}\\[0.1em]
vertreten durch \Anonym{Maria Musterfrau, LLM, Seppelgasse 12, 04711 Baumhaus, art21@example.com,}\\[0.1em]
--- Antragsgegner und Berufungsführer ---

wegen Anfechtung der Gründungsversammlung der Piratenpartei Deutschland

hat das Bundesschiedsgericht aufgrund der mündlichen Verhandlung vom 07.06.2013 durch die Richter Hens, Dempf und Hens-Dempf in der Sitzung am {\Datum} entschieden:

\begin{enumerate}
\item \textbf{Das Urteil des Landesschiedsgerichts Salzburg vom 1. April 2013, Az.: LSG-SB-03/13 wird aufgehoben.}
\item \textbf{Der Antrag auf Feststellung der Unwirksamkeit der Gründung der Piratenpartei Deutschland wird zurückgewiesen.}
\end{enumerate}

\section*{Sachverhalt}

..

\subsection{}

..

\subsection{}

..

\section*{Entscheidungsgründe}
..

\subsection{}

..

\subsection{}

..

\subsubsection{}
..

\subsubsection{}
..

\end{document}
