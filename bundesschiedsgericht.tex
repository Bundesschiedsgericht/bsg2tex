%
% bsg2tex - Bundesschiedsgericht
%
% Bundesschiedsgerichts-spezifische Anpassung von Seitenelementen
% Stand: 29.08.2014
%
% Andere Schiedsgerichte sollten diese Datei als Vorlage nehmen, und
% ihre eigene landesschiedsgericht.tex bauen und \include{}'n.
%

%
% Textblock oben rechts: Name, Anschrift, Mailadresse, Ort, Datum, Aktenzeichen
% Wenn die Gr��e der \parbox ge�ndert werden muss, wird eventuell eine Anpassung
% der Koordinaten und des freigestellten Bereichs im Template notwendig.
%
\newcommand{\Header}{
\parbox[t][36mm]{60mm}{\small%
Piratenpartei Deutschland\\
Bundesschiedsgericht\\
Pflugstra�e 9a, 10115 Berlin\\
schiedsgericht@piratenpartei.de\\[0.1em]
Berlin, den \textbf{\Datum}\\
AZ: \textbf{\Aktenzeichen}}
}

%
% Einleitungszeile des Footers
%
\newcommand{\RichterTabelleHeader}{%
\small\centering\textcolor{white}{%
Das Bundesschiedsgericht der Piratenpartei Deutschland wird vertreten durch:%
}%
}

%
% Der eigentliche Footer
% Eine 8-spaltige Tabelle (Anzahl 'c's in der Zeile mit 'tabular').
% In der einzeiligen Tabelle wird dann jede Spalte grunds�tzlich auf eine spezifische Breite (XXmm) forciert.
% F�r ein 7-k�pfiges Gremium werden 26mm empfohlen, wobei l�ngere Namen oder Titel ausnahmsweise
% breitere Spalten notwendig machen k�nnen ('Vorsitzender~Richter' bspw. 26.5mm).
% F�r ein 6-k�pfiges oder kleineres Gremium werden 28mm empfohlen.
% 
% Alle \parbox-Zeilen ausser der letzten m�ssen mit &% enden, und es sollten in dem ganzen Makro
% keine �berfl�ssigen Leerstellen oder Returns vorkommen.
%
% Soll die Tabelle insgesamt horizontal ausgebreitet werden, kann \tabcolsep (Spaltenabstand)
% von 0mm auf einen h�heren Wert gestellt werden. Einfach ausprobieren.
%
\newcommand{\RichterTabelle}{%
\setlength{\tabcolsep}{0mm}%
\centering\textcolor{white}{%
\large\begin{tabular}{ccccccc}%
\parbox[t]{26.5mm}{\centering Georg\\v. Boroviczeny\\[0.2em]\footnotesize{Ersatzrichter}}&%
\parbox[t]{26.5mm}{\centering Gregory\\Engels\\[0.2em]\footnotesize{Richter}}&%
\parbox[t]{26.5mm}{\centering Mario\\Longobardi\\[0.2em]\footnotesize{Richter}}&%
\parbox[t]{26.5mm}{\centering Michael\\Ebner\\[0.2em]\footnotesize{Vorsitzender~Richter}}&%
\parbox[t]{26.5mm}{\centering Klaus\\Sommerfeld\\[0.2em]\footnotesize{Richter}}&%
\parbox[t]{26.5mm}{\centering Holger\\van Lengerich\\[0.2em]\footnotesize{Richter}}&%
\parbox[t]{26.5mm}{\centering Stefan\\Thöni\\[0.2em]\footnotesize{Ersatzrichter}}%
\end{tabular}}%
}
